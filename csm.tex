\documentclass{amsart}

\usepackage{macros,slashed,amsrefs}

\def\ryan{\textcolor{green}{RG: }\textcolor{green}}
\def\brian{\textcolor{red}{BW: }\textcolor{red}}

\def\cbroid{{\bf C}}
\def\CA{\sC\sA}

\linespread{1.5}

\begin{document}

% \title[short text for running head]{full title}
\title{Higher Symplectic Sigma Models}



\section{Courant Sigma Model}

\subsection{Courant algebroids}

\begin{dfn}
A {\em Courant algebroid} on a manifold $X$ is a tuple $(E, \<-,-\>, a, \ll-,-\rr)$ where 
\begin{itemize}
\item[(i)] $E \to X$ is a (finite rank) vector bundle, whose sheaf of smooth sections we denote $\sE$;
\item[(ii)] $\<-,-\> : \sE \times \sE \to C^\infty_X$ is a nondegenerate, symmetric, bilinear pairing;
\item[(iii)] $a : \sE \to \sT_X$ is a $C^\infty_X$-linear map, called the anchor;
\item[(iv)] $\ll-,-\rr : \sE \times \sE \to \sE$ is a bilinear operator.
\end{itemize}
This data is required to satisfy the following equations. 
We denote $x,y,z \in \sE$, $f \in C^\infty_X$.
\begin{itemize}
\item[(1)] \brian{list}
\end{itemize}
\end{dfn}

Every Courant algebroid defines a Lie algebroid in the obvious way.
The conditions above imply that the tuple $(\sE, a, \ll-,-\rr)$ has the structure of a Lie algebroid on $X$. 

The collection of Courant algebroids on $X$ form a $1$-groupoid that we denote $\CA(X)$.
The objects are simply Courant algebroids, and the morphisms are bundle isomorphisms preserving the pairing, anchor, and bracket. 
Note that given any $U \subset X$ one has a natural map of groupoids $\CA(X) \to \CA(U)$ given by restriction.
The $1$-groupoid of Courant algebroids satisfies a nice gluing law with respect to affine charts on $X$. 
Namely, there is an equivalence of groupoids 
\ben
\CA(X) \simeq \lim_{U \subset X} \CA(U)
\een
where the limit is taken over the category of affine subsets of $X$. 

\begin{thm}[\cite{Roytenberg}] 
Let $L$ be a Lie algebroid.
There is a one-to-one correspondence between $2$-shifted (Roytenberg) symplectic structures on the dg manifold $[X/L] = (X, \cbroid^*(L))$ and Courant algebroid structures on $L$. 
\brian{say as equivalence of groupoids}
\end{thm}

If we relax the symplectic definition to include homotopy coherent $2$-shifted symplectic structures one finds the notion of a {\em twisted} Courant algebroid. 

There is a particularly well-behaved class of Courant algebroids that are important for us. 
First, note that the linear dual of the anchor map determines a map
\ben
a^* : \sT_X^\vee = \Omega^1_X \to \sE^\vee \cong \sE .
\een
In the isomorphism, we have identified $\sE$ with its dual via the pairing $\<-,-\>$. 
A Courant algebroid is {\em exact} if the resulting sequence of locally free sheaves
\ben
0 \to \Omega^1_X \xto{a^\vee} \sE \xto{a} \sT_X \to 0
\een
is exact. 
The above exact sequence determines a class in $H^1(X, \Omega^2_{cl})$. 

\begin{thm}\cite{Severa, SafPym, ...} 
The stack of exact Courant algebroids on $X$ is equivalent to the stack of of $1$-shifted closed two-forms $\Omega^2_{cl}(X) [1]$. 
In particular, every exact Courant algebroid is completely determined, up to isomorphism, by the class $[H] \in H^1(X, \Omega^2_{cl})$, called its "\v{S}evera" class.
\end{thm}


\subsection{The Courant $\sigma$-model}

\def\enh{{\rm enh}}

Every Courant algebroid defines the following dg Lie algebroid 
\ben
\left(\sT_X^\vee [1] \xto{a^\vee} \sE\right) \xto{a} \sT_X, 
\een 
that we denote by $\sL_{\sC}$. 
Here, the parentheses indicate the differential on the Lie algebroid, and $a$ is the anchor. 
By \cite{??} we know $(X, \enh(\sL_\sC))$ is an $L_\infty$-space equipped with a $2$-shifted symplectic structure that we denote by $\omega_{\sL}$. 

\begin{dfn}
Let $\sE$ be a Courant algebroid and $M$ a three-manifold. 
The perturbative Courant $\sigma$-model of maps from $M$ to $\sE$ near the smooth map $f : M \to X$ has underlying space of fields 
\ben
\Omega^*(M, f^*\enh(\sL_\sC))[1] .
\een
The $(-1)$-shifted pairing is defined by:
\ben
\<\alpha, \alpha'\> = \int \omega_{\sL} (\alpha \wedge \alpha') .
\een
The action functional is encoded by the local $L_\infty$-structure on $\Omega^*(M, f^*\enh(\sL_\sC))$. 
\end{dfn}

To begin, let's suppose that $f$ is a constant map. 
Then, as a graded $\Omega^\#_X$-module, the space of fields has the form
\ben
\Omega^\#(M) \tensor \Omega^\#(X, T_X^\vee [2] \oplus E[1] \oplus T_X) .
\een 

\ryan{Exact Courant algebroids are classified by their Severa class $H \in \Omega^3_{cl} (X)$.  Twisting the canonical topological boundary condition via this class may give a description of the $A$-model with $H$-flux $H$. See R. Szabo's work or the JHEP article of Bonechi--Cattaneo--Iraso (\ryan{they get it as a certain gauged fixing for the Poisson sigma model where the Poisson structure is the inverse of a Kahler form}).}

\subsection{Dirac Structures}

Suppose $\sE$ is a Courant algebroid (may or may not be exact) on $X$.

A Dirac structure is a subbundle $L \subset E$ such that 
\begin{itemize}
\item[(i)] $L$ is Largrangian with respect to the pairing $\<-,-\>$;
\item[(ii)] $L$ is involutive with respect to the bracket $[-,-]$.
\end{itemize}

There is a slightly modified version of a Dirac structure that is relative to a closed submanifold $i : Y \hookrightarrow X$. 
A Dirac structure on the pair $(\sE,Y)$ is a subbundle $L \subset f^*E$ such that 
\begin{itemize}
\item[(i)] $L$ is Largrangian with respect to the pairing $\<-,-\>$;
\item[(ii)] $L$ is compatible with the anchor, in the sense that $a(L) \subset TY \subset f^* TX$. 
\item[(ii)] $L$ is involutive with respect to the bracket $[-,-]$.
\end{itemize}

\brian{Pavel classifies Lagrangian structures on morphisms $[Y / \sM] \to [X / \sE]$ where $\sE$ is a Courant algebroid so $[X/ \sE]$ is $2$-symplectic. 
In the case $Y = X$ you get the first type of Dirac structures. 
In the general case, you get the second type. 
}

In the exact case.

\begin{prop} 
Suppose we use the Dirac structure for the standard exact Courant algebroiod $\sE = T_X \oplus T_X^\vee$ defined by a Poisson structure $(X, \Pi)$. 
The corresponding boundary theory for the Courant $\sigma$-model of maps from $\Sigma \times \RR_{\geq 0}$ is equivalent to the Poisson $\sigma$-model on $\Sigma$ with target $(X, \Pi)$. 
\end{prop}


\subsection{Link Invariants}

Let $E$ be a Courant algebroid and $R$ a representation up to homotopy of the associated 2-symplectic Lie algebroid.  Further, assume that $R$ is equipped with an invariant trace.  Wilson loop observables determine invariants for links in a 3-manifold source manifold (anomalies?).

\brian{I claim there are no anomalies for any Courant $\sigma$-model. This should probably wait till a later paper, but perhaps it's good for us to study the local deformation complex still...} 


\end{document}